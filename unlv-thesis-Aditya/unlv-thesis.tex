%if printing on both sides of a page add 'twopage' to the [...] below
\documentclass[11pt,openright]{report} 
\usepackage{graphicx}
\usepackage{color}
\usepackage{tocbibind}
%\usepackage{algorithm2e} % must be included before unlv-thesis
\usepackage[linesnumbered,ruled]{algorithm2e}
\SetKwProg{Fn}{Function}{}{}
\usepackage{subcaption}
\usepackage{fancyhdr}
\usepackage{unlv-thesis}

\graphicspath{{./images/}, {./results/}}
\usepackage{hyperref}
\hypersetup{
	colorlinks=false, %set true if you want colored links
	linktoc=all,     %set to all if you want both sections and subsections linked
	linkcolor=blue,  %choose some color if you want links to stand out
}

\setcounter{tocdepth}{3}% Include \subsubsection in ToC
\setcounter{secnumdepth}{3}% Number \subsubsection

%%%
%%% Choose either \phdthesis or \mastersthesis
\mastersthesis
%\phdthesis

%%%
%%% opens all chapters on right hand sides (needed for double sided printing)
\rightchapter
%%% add \twosided if you are printing on both sides
%\twosided
%%%
%%% Choose the spacing for the thesis: \singlespace, \oneandhalfspace or \doublespace
%\oneandhalfspace
%\singlespace
\doublespace

\makeindex

%%%
%%% The name of your thesis and your own name. Title must be in all
%%% caps and in an inverted triangle
\thesistitle{A MACHINE LEARNING APPROACH TO PREDICT FIRST-YEAR STUDENT RETENTION RATES AT UNIVERSITY OF NEVADA, LAS VEGAS}
\thesistitlelowercase{A Machine Learning Approach to Predict First-Year Student Retention Rates at University of Nevada, Las Vegas  }
\thesisauthor{Aditya Rajuladevi}

%%%
%%% Add your previous degrees here
\thesisauthorpreviousdegrees{
Bachelor Degree in Computer Engineering\\ \vspace*{-0.12in}
Jawahar Lal Nehru Technological University, Hyderabad, India\\ \vspace*{-0.12in}
2014}

%%%
%%% Month and Year to appear on the thesis
\thesismonth{May} 
\thesisyear{2018}
\copyrightyear{2018}

%%%
%%% The size of the committee: chair + other members + college rep (normally 4)
%%% use \chair{}, \memberone{}, \membertwo{}, \memberthree{}, \colleferep{}
\thesiscommitteesize{4}
%\signatures{}   % will generate 'signatures on the approval page' grad college does 
%not like this, so don't do that in their version
\chair{Fatma\ Nasoz, Ph.D.}
\memberone{Laxmi\ Gewali, Ph.D.}
\membertwo{Justin\ Zhan, Ph.D.}
\collegerep{Magdalena\ Martinez, Ph.D.}

%----------------------- Macros ----------------------------------------- 

\include{my_macros}

%---------------------- Thesis starts here ------------------------------

% The organization should be as follows, as per our online guidelines:
% page (not numbered): title page
% page (not numbered): copyright statement (this page is optional)
% page ii: approval page, but do not inlcude it until the grad college says ok.
% page iii:  abstract
% acknowledgments
% preface
% table of contents
% list of tables 
% list of figures
% page 1 -> ???: main body of text
% exhibits (what ever that is)
% appendices
% bibliography,
% author's CV. 

% everything after main body should have regular page numbers, 
% everything before should have roman numeral in small letters.
\begin{document}
\thesistitlepage
\copyrightpage

\newpage
%% here goes the approval page - uncomment the following line when you 
%% get an ok from the graduate college:
%%
%% \approvalpage for the page that people need to sign
%%
%% \electronicapprovalpage for the page that needs to be used when submitting the PDF

%\approvalpage
\electronicapprovalpage


\begin{thesisabstract}
First-Year student retention rates refer to the percentage of first-year students who return to the same institution for their sophomore year. The national average in the institutions at the U.S for the year 2016 is at around 77 \% which indicates that most of the universities are performing poorly in terms of retaining the first-year students. First-year retention rates act as an important indicator of the student satisfaction as well as the performance of the university. Moreover, universities with low retention rates may face a decline in the admissions of talented students with a notable loss of tuition fees and contributions from alumni. Hence it became important for universities to formulate strategies to identify students at risk and take necessary measures to retain them. Many universities have tried to develop successful intervention programs to help students increase their performance. However, identifying and prioritizing students who need early interventions still remains to be very challenging.  

The retention rate at the University of Nevada, Las Vegas (UNLV) is close to 74\% which indicate the need for specific intervention programme's to retain the students at risk of dropping out after their first year. In this thesis, we propose the use of predictive modeling methods to identify students who are at risk of dropping out after their first year at an early stage to whom the instructors can offer help. We apply several important classification algorithms from machine learning such as Logistic Regression, Decision trees, and Random forest classifier and evaluate them using metrics useful to the educators at UNLV. We use feature selection methods to identify important variables used in each model to increase the generalization and accuracy of the predictions. We propose to design effective metrics to evaluate the model's performance to help match at-risk students with appropriate supports. The main focus of this research is on students at risk of not being retained at the end of the first year, but it also lays a foundation for future work on other adverse academic outcomes. 
\end{thesisabstract}


%%% if you have a preface it should go here before the acknowledgement


%%%
%%% Here goes the acknowledgements if you have any - if none then delete or
%%% comment out completely
\begin{thesisacknowledgments}
I would like to express my sincere gratitude to my advisor, Dr. Fatma Nasoz, for her motivation, guidance, and support throughout the research. She continuously steered me in the right direction in this research as well as my Master's program.

I would also like to extend my thanks to Dr. Laxmi Gewali, Dr. Justin Zhan, and Dr. Magdalena Martinez for their support and for being a part of my thesis committee. I am really grateful for all the support from Dr. Ajoy K Datta who was always available to me whenever I needed his guidance.

I am gratefully indebted to Kivanc Oner, Carrie Trentham and Becky Lorig from the Enterprises Application Services department at UNLV for their continuous support and valuable comments on this thesis. They answered my many questions about student enrollments and retention problems at UNLV and played a major role in helping me find the student data I was looking for from the UNLV data warehouse.

My deep sense of gratitude to my parents Venkat Rao Rajuladevi, Mallika Rajuladevi and my sister Arthi Rajuladevi who are my moral strength and motivation. I would like to thank Sai Phani Krishna Parsa and Paritosh Parmar for their constant support and guidance throughout my Master's program.

Finally, I would like to thank all my friends, seniors and juniors who made my time here at UNLV very memorable. 
\end{thesisacknowledgments}

%%%
%%% Magic. If you remove this you will not get page numbers on 
%%% any 2nd, 3rd or so on page of any of the tableofcontents
%%% or listofXXXX
%%%
\fancypagestyle{plain}{%
  \fancyhf{}
\renewcommand{\headrulewidth}{0.0pt}
\renewcommand{\footrulewidth}{0.0pt}
  \fancyfoot[C]{\thepage}
}
\pagestyle{plain}

%%%
%%% If you don't want list of figure and list of tables then just put a % 
%%% in front of the two lines here:
\tableofcontents
\clearpage
\listoftables
\clearpage
\listoffigures
\clearpage
% if you want a list of algorithms, make sure to use the Makefile-loa instead of Makefile.
\listofalgorithmes
\clearpage

%%%
%%% This is the start of the first 
\chapter{Introduction}\label{chapter:introduction} 

The first-year or freshmen retention rate refers to the number of freshmen in a college or university who return for their sophomore year. Many universities are facing huge problems with low or decreasing first-year student retention rates. Low retention rates are a bad indicator of the university's performance and can damage the reputation of the institution in the eyes of students and parents. The reasons behind student dropout after the first year in universities can range from high expectations of the college programs, transition into an interdisciplinary curriculum, economic problems, inability to mix well with other students or struggling due to unfulfilled prerequisite requirements \cite{lau2003institutional}. Many researchers have formulated solutions such as building learning communities, providing additional resources \cite{tinto1999taking}, highlighting student participation in campus life and providing academic support \cite{lau2003institutional}. Also, few studies have indicated that the risk of dropping out decreases with an increase in academic performance \cite{Murtaugh}. Thus, one way to increase retention is to increase academic success. In recent years, many universities have invested significantly in development and implementation of intervention programs to help at-risk students and support them individually to improve their academic performance. 

The success of such intervention programs depends on the university's ability to accurately identify students who need help. In a traditional approach, many universities have used academic performance indicators such as GPA's, absence rates, previous grades, SAT or ACT scores from enrollment data to generate rules that can be used to identify students at risk \cite{bingham2016}. Although such rule-based systems served as good indicators of identifying at-risk students for some years, they had some downsides such as fewer accuracies, static,  expensive to generate and maintain and most importantly they lacked a validation mechanism to verify the predictions. Alternatively, recent research has indicated the potential value of machine learning algorithms such as Logistic Regression, Random Forest Classifiers, Decision Trees, Support Vector Machines (SVM) and Neural networks for the problem \cite{plagge2013using,lakkaraju2015machine,marbouti2016models}. These algorithms when trained using traditional academic data can identify at-risk students more accurately. The performance of these algorithms can be evaluated using various metrics such as Precision, Recall, and Area Under Curve (AUC) thus giving us a good indicator to validate the results. However, the application of such predictive methods to identify at-risk students is still at its early stages, owing to the implementation complexity and the availability of data. Currently, many universities have defined rules in the collection of data to use for such a research.

Over the recent years, the retention rates at UNLV have displayed a highly varying pattern. It dropped from 77\% in 2012 to 74\% in 2014 and then increased to 77\% in 2015, which later on fell to 74\% in 2016 Figure \ref{fig:unlv_retention_trend}. Such an unstable pattern of freshmen retention rates has drawn a lot of attention by the educators and administration at UNLV. Hence, a predictive approach to identifying at-risk students and supporting them with additional resources would be quite beneficial to increase the retention rates at UNLV. 

\begin{figure}
	\centering
	\includegraphics[scale=0.7]{unlv-retentiontrend}
	\caption{Freshmen Retention Rates at UNLV}
	\label{fig:unlv_retention_trend}
\end{figure}

\section{Objective}\label{section:objective}
The objective of this thesis is to create predictive models that can be used to identify at-risk students. In this thesis, important machine learning algorithms such as Logistic Regression, Decision trees, Random Tree Classifiers and SVM's will be trained on real-time student data obtained from UNLV's enrollment census. The trained models will then be used to predict at-risk students from a test dataset which the model has not seen earlier. The models are evaluated and compared using metrics such as precision, recall, and area under the curve to determine which model provides the best results. The results of the analysis such as the evaluation metrics will be converted to risk scores which can be easily understood by the educators and administrators at UNLV. Another contribution of this thesis is to rank students based on risk scores which will be very helpful in an efficient allocation of resources as part of the intervention programs.

\section{Outline}\label{section:outline}

In Chapter \ref{chapter:introduction}, the brief topic of First-Year student retention rates, its importance to universities and the proposed approach to increase the retention rates at UNLV was described.\newline

\noindent In Chapter \ref{chapter:background}, we will discuss the existing research on improving first-year retention rates and the background information required to understand the proposed predictive approach using machine learning. It will also cover the most important and popular algorithms of machine learning.
\newline

\noindent In Chapter \ref{chapter:methodology}, we will describe the methodology adapted for the analysis. 
\newline

\noindent In Chapter \ref{chapter:experiment_results}, we will present the experimental results. The characteristics of datasets used for these methods will also be described.
\newline

\noindent In Chapter \ref{chapter:conclusion}, we will summarize the proposed methods and their results along with the possible extension of this thesis.

\chapter{Background and Preliminaries} \label{chapter:background}
\section{Related Work}\label{section:relatedwork}

The prediction of first-year student retention rates and identification of students at risk of not being retained has been a well-researched problem in the area of higher education sector for decades. Early studies involved learning the important factors that lead to student dropout by developing a theoretical model. Tinto is one of the major and earliest researchers in this area. Tinto's student engagement model \cite{tinto1999taking} has served as the basis for a large number of theoretical studies \cite{braxton2002introduction}. Similar research was carried out by Ernest Pascarella, Patrick Terenzini, and Alexander Astin, which focused more on the external factors such as the institution's administration and its policies when determining the reasons for student retention \cite{astin2012assessment}. Tinto in his 2006 study \cite{tinto2006} has stated that there has been a huge increase in the number of businesses and organizations to analyze and help institutions with the student retention problem. Later, in the same study, he revealed that there was only little change in the retention rates even with some huge businesses helping the universities. He also described the importance of external factors such as student-faculty relationships, extracurricular program, and orientation programs for first years. Moreover, he incorporated the role of academic factors into his model to make it more suitable to the college structure \cite{tinto2006}. Astin in his Input-Environment model \cite{astin2012assessment}, suggests that researchers should consider pre-college factors such as gender, race/ethnicity, family background, high school GPA  as important for student retention.

In addition to understanding the factors responsible for student dropout, the researchers were interested in identifying students at risk of not being retained in order to intervene and prevent them from dropping out. Early research included usage of statistical and analytical methods such as logistic regression and discriminant analysis for predicting student retention rates\cite{lakkaraju2015machine,marbouti2016models,adejo2017}. The results from these models showed that the learning algorithms were in fact better than many existing rule based models in learning patterns from the existing student data. Educational data mining has emerged into an important field of research in studying student retention, because of its high accuracy and robustness in working with missing data been\cite{alkhasawneh2014developing}. In another study, Jay Bainbridge, James Melitski, Anne Zahradnik, Eitel J. M. Lauría, Sandeep Jayaprakash and Josh Baron used fall 2010 undergraduate students data from four different sources and applied classifiers such as logistic regression, support vector machines and c4.5 decision trees for prediction and comparison purposes\cite{bainbridge2015}. The results showed that logistic regression and SVM trees provided higher classification accuracies compared to the decision trees to predict students at risk. 

Serge Herzog a researcher from University of Nevada, Reno(UNR) campus has done some extensive research on student retention and graduate prediction. He used Decision trees and Neural networks to predict student retention of data from UNR\cite{herzog2006estimating}. FarshidMarbouti, Heidi A.Diefes-Dux, KrishnaMadhavan \cite{marbouti2016models} have compared seven different prediction models for identifying at-risk students using in-semester performance factors (i.e., grades) and based on standards-based grading.Another similar research focused on the problem of imbalanced output class distribution in the field of student retention in which the researchers tested three balancing techniques such as over-sampling, under-sampling and synthetic minority over-sampling (SMOTE) along with machine learning algorithms.

Although predictive analysis using machine learning models have proved to be very effective in identifying at-risk students, they were still not efficient and useful to educators who wanted to develop academic support programs and resources to support the identified students. The primary reason being the lack of understanding of the metrics from the predictive models by the educators. A few researchers analyzed this issue at the high school level and came up with a framework to convert model accuracies to risk scores that could be used by the educators in the efficient allocation of the resources \cite{lakkaraju2015machine}. Though the above-mentioned framework was giving good results, it was restricted to school level data and thus cannot yield accurate results in the university level as both are very different environments with different set of factors. Hence, in this thesis we try to include such an analysis into college level studies of UNLV and compare our results to the existing approaches.


\section{Preliminaries}\label{section:preliminaries}

\subsection{Machine Learning Concepts}

\noindent  Tom Mitchell defined Machine Learning as \cite{Mitchell1997}: 
\newline\newline
\hangindent=0.7cm "A computer is said to learn from experience E with respect to some class of tasks T and performance measure P, if its performance at tasks in T, as measured by P, improves with experience E." \newline 

\noindent Example: playing chess.

\noindent E = the experience of playing many games of chess with different people

\noindent T = the task of playing chess.

\noindent P = the probability that the program will win the next game.\newline 

\noindent In general machine learning tries to learn patterns inherent in the underlying data and remembers it as experience, which it uses for predictions. It was conceptualized from the notion of learning process adopted by a human brain. Just as the human brain gets better at a specific task by repeated learning and previous experiences, a computer will also learn more patterns hidden in the data based on its previous experiences. Additionally the huge processing power of computers enables them to perform such a learning process in identifying patterns from complex data which can be very difficult for a human to understand.

\subsection{Predictive Analytics}
Predictive analytics primarily deals with extracting patterns using machine learning models from existing datasets. the extracted patterns are used to predict future events and behaviors in previously unseen data. Predictive analytics is being used in wide range of fields such as education, finance, automobile, and healthcare. The analytics is performed by running different machine learning algorithms on previously collected data. The algorithms try to learn patterns between different properties of the data and preserve the knowledge in model parameters. The resulting model is able to predict the unknown property of a future unseen data.

 \begin{table}[!t]
	\renewcommand{\arraystretch}{1.3}
	\caption{Admissions data at a University}
	\label{table:example_db}
	\centering
	\begin{tabular}{|c|c|c|c|c|}
		\hline
		\bfseries Age & \bfseries Gender & \bfseries HSGPA & \bfseries ACT & \bfseries Accepted\\
		\hline
		$20$ & Male & 3.89 & 25.6 & 1\\ \hline
		$21$ & Female & 3.20 & 22.6 & 1\\ \hline
		$20$ & Female & 2.50 & 18.3 & 0\\ \hline
		$22$ & Male & 3.10 &  19.2 & 0\\ \hline
		$19$ & Female & 3.60 & 23.5 & 1\\ \hline
	\end{tabular}
\end{table}



An illustrative example is shown in Table \ref{table:example_db} which shows student data captured during admissions at a university. The aim is to predict if the student will be accepted or not by looking at the other variables in the data. The column "Accepted" is said to be a dependent variable, and all the other variables are called independent variables. A "1" in the "Accepted" column represents the outcome of the student being accepted into the university and a "0" means he was rejected. To this data we apply a machine learning algorithm, that will learn a prediction model based on the relations between the input independent variables and the output dependent variable. The learned model is then able to predict the output variable by taking other variables as input.
  \begin{figure}
	\centering
	\includegraphics[scale=0.5]{MLApproachFinal}
	\caption{Machine Learning Approach}
	\label{fig:predictive_analysis-approach}
\end{figure}
 

The general approach for applying predictive analysis on a data can be divided into 2 phases as shown in figure \ref{fig:predictive_analysis-approach}. Build phase deals with the process of creating a prediction model and testing its performance. The process of creating the prediction model is known as training and the data used is known as training data. To test the effectiveness of the created model, it is tested on another set of previously unseen data known as testing data. We use two different data sets to check the generalization capacity of the model in predicting previously unseen data. If we use all the available data for training, the model may learn too much from the inputs giving rise to the problem of overfitting. Overfitting occurs when a model is performing well on its training data, but not on the other unseen data. A common approach to handle the problem of overfitting is to divide the input data into training and testing sets. The model is then evaluated for performance using test data.  The evaluated model is then used in the operational phase to perform predictions given a new data. The metrics used to evaluate a model are explained in detail in coming chapters.


\subsection{Selected Models}

In this section we discuss various models that were used in this thesis to analyze the student data from UNLV. The models selected for comparison were Logistic Regression, Decision Trees, Random Forest, and SVM's.

\subsubsection {Logistic Regression Model}

\subsubsection {Decision Trees}

\subsubsection {Random Forest Classifiers}

\subsubsection {Support Vector Machines }

\subsection {Evaluation Methods}

\chapter{Methodology} \label{chapter:methodology}

\section{Data Collection}

The Operations Support and Reporting Office at UNLV is responsible for extracting information regarding student enrollment and performance from the UNLV Data Warehouse and reporting the census to Integrated Postsecondary Education Data System (IPEDS). The data used in this thesis were obtained from the census data submitted to IPEDS every year. The following subsections will describe more information about the data.

\section {Data Pre-Processing}
\subsection {Computation of Not Retained variable}
How did you generate the output variable
\subsection {Handling Outliers in Data}

Generally the process of collecting data from different sources can introduce outliers into the dataset. This may be due to a faulty data extraction program or a human error. In predictive data analysis outliers can have significant effect on the predictive power of the models as they can introduce huge bias into the model. Hence it is almost always essential to check for outliers in the dataset before performing any analysis. Outliers were identified in our dataset. Data plots with index on x-axis and values on y-axis was used to demonstrate the presence of outlier in a feature.
\subsubsection{UnwHSGPA scores}





\begin{figure}
\centering
    \begin{subfigure}[b]{0.55\textwidth}            
            \includegraphics[width=\textwidth]{UnwHSGPAScores}
            \caption{UnwHSGPA with Outliers}
            \label{fig:UnwHSGPA-with-outliers}
    \end{subfigure}%
     %add desired spacing between images, e. g. ~, \quad, \qquad etc.
      %(or a blank line to force the subfigure onto a new line)
    \begin{subfigure}[b]{0.55\textwidth}
            \centering
            \includegraphics[width=\textwidth]{UnwHSGPAWithoutScores}
            \caption{UnwHSGPA without Outliers}
            \label{fig:UnwHSGPA-No-outliers}
    \end{subfigure}
    \caption{Removing Outliers from UnwHSGPA}\label{fig:unwHSGPA}
\end{figure}

 There were a few outliers identified for variable UnwHSGPA, as shown in figure \ref{fig:UnwHSGPA-with-outliers}. The outliers of the UnwHSGPA were replaced by mean value of the said dataset excluding all the outliers. Data plot of the variable UnwHSGPA after removing outliers is shown in figure \ref{fig:UnwHSGPA-No-outliers}
\subsection {Handling Missing Values}
Explain your approach of missing values
\subsection {Data Selection and Transformations}
What important data did you select for your analysis
\subsection {Correlation of Features}
check the correlation of factors


\chapter{Experimental Results} \label{chapter:experiment_results}
\section {Analysis of Predictive Models}
\subsection {Evaluation using Traditional Metrics}
\chapter{Conclusion} \label{chapter:conclusion}
%%%
%%% The bibliography - you can change the alpha to suit your style if you don't like it is it is.
%%% The bib file should be called 'thesis.bib' - if not then change the second line here to be correct.
\bibliographystyle{alpha}
\thesisbibliography{thesis}


%%%
%%% Vita comes next
\vita
\chapter{} %% please leave this one blank - the vita stuff is sort of a hack.
\linespread{1.3} 
\begin{center}
Graduate College\\
University of Nevada, Las Vegas\\[1cm]
Aditya Rajuladevi\\[1cm]
\end{center}

\noindent Degrees:\\
\indent Bachelor Degree in Computer Engineering 2014\\
\indent Jawaharlal Nehru Technological University, Hyderabad, India\\

\noindent Thesis Title: A Machine Learning Approach to Predict First-Year Student Retention Rates at University of Nevada, Las Vegas\\

\noindent Thesis Examination Committee:\\
\indent Chairperson, Dr. Fatma Nasoz, Ph.D.\\
\indent Committee Member, Dr. Laxmi Gewali, Ph.D.\\
\indent Committee Member, Dr. Justin Zhan, Ph.D.\\
\indent Graduate Faculty Representative, Dr. Magdalena Martinez, Ph.D.\\

\end{document}





